\documentclass[10pt,a4paper]{article}
\usepackage{tikz}
\usetikzlibrary{matrix,calc}
\usepackage{multirow}
\usepackage[utf8]{inputenc}
\usepackage[spanish]{babel}
\usepackage{amsmath}
\usepackage{amsfonts}
\usepackage{amssymb}
\usepackage{graphicx}
\usepackage{float}
\usepackage[left=2cm,right=2cm,top=2cm,bottom=2cm]{geometry}
\usepackage{circuitikz}
\usepackage{tikz-timing}
\usetikztiminglibrary[rising arrows]{clockarrows}
\usepackage{pgfplots}
\pgfplotsset{compat=1.15}
\usepackage{listings}
\usepackage[export]{adjustbox}
\usepackage{graphicx,wrapfig,lipsum}
\usepackage{booktabs}
\usepackage[skip=0pt]{caption}
\usepackage{pdfpages}
\captionsetup[figure]{font=small,labelfont=small}


\input{karnaugh_def}
\newenvironment{KarnaughTP3}%
{
\begin{tikzpicture}[baseline=(current bounding box.north),scale=0.8]
\draw (0,0) grid (4,4);
\draw (0,4) -- node [pos=0.9,above right,anchor=south west] {$Q_1 Q_0$} node [pos=0.7,below left,anchor=north east] {$W Q_2$} ++(135:1);
%
\matrix (mapa) [matrix of nodes,
        column sep={0.8cm,between origins},
        row sep={0.8cm,between origins},
        every node/.style={minimum size=0.3mm},
        anchor=8.center,
        ampersand replacement=\&] at (0.5,0.5)
{
                       \& |(c00)| 00         \& |(c01)| 01         \& |(c11)| 11         \& |(c10)| 10         \& |(cf)| \phantom{00} \\
|(r00)| 00             \& |(0)|  \phantom{0} \& |(1)|  \phantom{0} \& |(3)|  \phantom{0} \& |(2)|  \phantom{0} \&                     \\
|(r01)| 01             \& |(4)|  \phantom{0} \& |(5)|  \phantom{0} \& |(7)|  \phantom{0} \& |(6)|  \phantom{0} \&                     \\
|(r11)| 11             \& |(12)| \phantom{0} \& |(13)| \phantom{0} \& |(15)| \phantom{0} \& |(14)| \phantom{0} \&                     \\
|(r10)| 10             \& |(8)|  \phantom{0} \& |(9)|  \phantom{0} \& |(11)| \phantom{0} \& |(10)| \phantom{0} \&                     \\
|(rf) | \phantom{00}   \&                    \&                    \&                    \&                    \&                     \\
};
}%
{
\end{tikzpicture}
}

\newenvironment{KarnaughvuiteTP3}%
{
\begin{tikzpicture}[baseline=(current bounding box.north),scale=0.8]
\draw (0,0) grid (4,2);
\draw (0,2) -- node [pos=0.9,above right,anchor=south west] {$Q_1Q_0$} node [pos=0.7,below left,anchor=north east] {$W$} ++(135:1);
%
\matrix (mapa) [matrix of nodes,
        column sep={0.8cm,between origins},
        row sep={0.8cm,between origins},
        every node/.style={minimum size=0.3mm},
        anchor=4.center,
        ampersand replacement=\&] at (0.5,0.5)
{
                      \& |(c00)| 00         \& |(c01)| 01         \& |(c11)| 11         \& |(c10)| 10         \& |(cf)| \phantom{00} \\
|(r00)| 0             \& |(0)|  \phantom{0} \& |(1)|  \phantom{0} \& |(3)|  \phantom{0} \& |(2)|  \phantom{0} \&                     \\
|(r01)| 1             \& |(4)|  \phantom{0} \& |(5)|  \phantom{0} \& |(7)|  \phantom{0} \& |(6)|  \phantom{0} \&                     \\
|(rf) | \phantom{00}  \&                    \&                    \&                    \&                    \&                     \\
};
}%
{
\end{tikzpicture}
}

\begin{document}
\section{Ejercicio 2}

\subsection{Moore}
Se nos pidi\'o realizar una m\'aquina de Moore y una de Mealy que puedan detectar la secuencia 1-1-0-1 y avise en la salida al detectarla. Para esto tuvimos en cuenta que cuando termina la secuencia y es detectada se toma el \'ultimo 1 de la secuencia como el primero de la siguiente secuencia en caso de que ocurran 2 secuencias seguidas, la cual ser\'ia 1-1-0-1-1-0-1. Primero diseñamos un diagrama de estados basándonos en la m\'aquina de Moore, donde tendremos 5 estados. El estado A es el caso base donde se recibió un 0 fuera de la secuencia pedida, el estado B es el caso donde se recibió el primer 1, el estado C es el caso donde se recibe el segundo 1, el estado D donde se recibe la secuencia 1-1-0 y el estado E es cuando se recibe la secuencia completa. 
El diagrama queda de la siguiente manera:

\begin{figure}[hbtp]
	\centering
		\includegraphics[scale=1]{Imagenes/diagestmoore.jpg}
	\caption{Diagrama de Estados - Máquina de Moore}
	\label{2_fig0}
\end{figure}

Se le otorgó a cada estado un valor representativo en binario, siendo el estado A un 0 y el estado E un 6 (el cual al estar en binario se representara como 1-1-0), no utilizamos los valores comunes del 0 al 4 ya que preferimos utilizar el código de Grey para que los flip flops tuvieran menos transiciones al pasar de un estado al otro. Como el estado E es un número que necesita 3 bits de memoria, se utilizar\'an 3 flip flop para almacenar el número del estado actual, cada flip flop se representar\'a en este ejercicio como $Q_n$, siendo entonces cada uno representación de un bit del estado en el cual se encuentra el circuito. El flip flop $Q_0$ representa el bit menos significativo, y el $Q_2$ el bit más significativo. Cada estado podr\'a pasar a otro seg\'un la entrada que reciba, como se mostró en el diagrama anterior, ahora pasamos a representar el esquema en una tabla con las variaciones de los estados:

\begin{figure}[H]
	\begin{center}
		\begin{tabular}{c|c|c|c|}
		\cline{2-4}
		\textbf{} & \textbf{W = 0} & \textbf{W = 1} & \textbf{Z} \\ \hline
		\multicolumn{1}{|c|}{\textbf{A$_{000}$}} & \textbf{A} & \textbf{B} & \textbf{0} \\ \hline
		\multicolumn{1}{|c|}{\textbf{B$_{001}$}} & \textbf{A} & \textbf{C} & \textbf{0} \\ \hline
		\multicolumn{1}{|c|}{\textbf{C$_{011}$}} & \textbf{D} & \textbf{A} & \textbf{0} \\ \hline
		\multicolumn{1}{|c|}{\textbf{D$_{010}$}} & \textbf{A} & \textbf{E} & \textbf{0} \\ \hline
		\multicolumn{1}{|c|}{\textbf{E$_{110}$}} & \textbf{A} & \textbf{C} & \textbf{1} \\ \hline
		\end{tabular}
	\caption{Transiciones con Estados - Máquina de Moore} 
	\label{2_fig1}
	\end{center}
\end{figure}
Si ahora representamos a cada estado con sus respectivos valores $Q_n$ para ver las transiciones la tabla quedar\'a con valores 1 y 0 que representar\'an una salida High o Low respectivamente. As\'i podremos analizar cada flip flop por separado y llegar a un circuito combinacional que los alimente, para esto tenemos que discriminar entre los estados actuales $Q{n_{t}}$ y los estados siguientes $Q{n_{t+1}}$. La tabla dicha es la siguiente:


\begin{figure}[H]
	\begin{center}
		\begin{tabular}{|c|c|c|c|c|c|c|c|c|c|}
\hline
\multicolumn{3}{|c|}{\multirow{2}{*}{\textbf{Estado Actual}}} & \multicolumn{6}{c|}{\textbf{Estado}} & \multirow{2}{*}{\textbf{Salida}} \\ \cline{4-9}
\multicolumn{3}{|c|}{} & \multicolumn{3}{c|}{\textbf{W = 0}} & \multicolumn{3}{c|}{\textbf{W = 1}} &  \\ \hline
\textbf{$Q_{2_{t}}$} & \textbf{$Q_{1_{t}}$} & \textbf{$Q_{0_{t}}$} & \textbf{$Q_{2_{t+1}}$} & \textbf{$Q_{1_{t+1}}$} & \textbf{$Q_{0_{t+1}}$} & \textbf{$Q_{2_{t+1}}$} & \textbf{$Q_{1_{t+1}}$} & \textbf{$Q_{0_{t+1}}$} & \textbf{Z} \\ \hline
\textbf{0} & \textbf{0} & \textbf{0} & \textbf{0} & \textbf{0} & \textbf{0} & \textbf{0} & \textbf{0} & \textbf{1} & \textbf{0} \\ \hline
\textbf{0} & \textbf{0} & \textbf{1} & \textbf{0} & \textbf{0} & \textbf{0} & \textbf{0} & \textbf{1} & \textbf{1} & \textbf{0} \\ \hline
\textbf{0} & \textbf{1} & \textbf{1} & \textbf{0} & \textbf{1} & \textbf{0} & \textbf{0} & \textbf{1} & \textbf{1} & \textbf{0} \\ \hline
\textbf{0} & \textbf{1} & \textbf{0} & \textbf{0} & \textbf{0} & \textbf{0} & \textbf{1} & \textbf{1} & \textbf{0} & \textbf{0} \\ \hline
\textbf{1} & \textbf{1} & \textbf{0} & \textbf{0} & \textbf{0} & \textbf{0} & \textbf{0} & \textbf{1} & \textbf{1} & \textbf{1} \\ \hline
\textbf{1} & \textbf{1} & \textbf{1} & \textbf{X} & \textbf{X} & \textbf{X} & \textbf{X} & \textbf{X} & \textbf{X} & \textbf{X} \\ \hline
\textbf{1} & \textbf{0} & \textbf{1} & \textbf{X} & \textbf{X} & \textbf{X} & \textbf{X} & \textbf{X} & \textbf{X} & \textbf{X} \\ \hline
\textbf{1} & \textbf{0} & \textbf{0} & \textbf{X} & \textbf{X} & \textbf{X} & \textbf{X} & \textbf{X} & \textbf{X} & \textbf{X} \\ \hline
		\end{tabular}
		\caption{Transiciones con Flip Flop - Máquina de Moore} 
		\label{2_fig2}
	\end{center}
\end{figure}

Para analizar esta tabla debemos tener en cuenta que para la máquina de Moore los estados son dependientes de las entradas y de ellos mismos, por lo que cada estado $Q_{n_{t}}$ depender\'a tanto de la entrada $W$ como de los estados $Q_{2_{t}}$,$Q_{1_{t}}$ y $Q_{0_{t}}$. Mientras que la salida $Z$ depende solo de los estados $Q_{2_{t}}$, $Q_{1_{t}}$ y $Q_{0_{t}}$. Por lo que pasaremos a analizar cada columna $Q_{n_{t+1}}$ dependiendo de cada combinaci\'on $Q{n_{t}}$ y la entrada $W$, y luego analizaremos la columna $Z$ para cada combinaci\'on de los $Q{n_{t}}$. Para analizar las columnas las resolvimos con mapas de Karnaugh para simplificar más rápido los minit\'erminos quedando los estados y la salida de las siguientes maneras:
	
\begin{figure}[H]
	\begin{center}
	\begin{KarnaughTP3}
	\contingut{0,0,0,0,X,X,0,X,1,1,0,1,X,X,1,X}
	\implicant{12}{14}{red}
	\implicant{12}{9}{blue}
	\implicant{13}{11}{green}
	\end{KarnaughTP3}
	\end{center}
	\caption{$Q_{0_{t+1}}$ Máq. de Moore} 
	\label{2_fig3}
\end{figure}

\begin{figure}[H]
	\begin{center}
		\begin{KarnaughTP3}
		\contingut{0,0,0,1,X,X,0,X,0,1,1,1,X,X,1,X}
		\implicant{12}{14}{orange}
		\implicant{13}{11}{green}
		\implicant{3}{11}{red}
		\implicant{15}{10}{blue}
		\end{KarnaughTP3}
	\end{center}
	\caption{$Q_{1_{t+1}}$ Máq. de Moore} 
	\label{2_fig4}
\end{figure}

\begin{figure}[H]
	\begin{center}
		\begin{KarnaughTP3}
		\contingut{0,0,0,0,X,X,0,X,0,0,1,0,X,X,0,X}
		\implicant{10}{10}{red}
		\end{KarnaughTP3}
	\end{center}
	\caption{$Q_{2_{t+1}}$ Máq. de Moore} 
	\label{2_fig5}
\end{figure}
	
\begin{figure}[H]
	\begin{center}
		\begin{KarnaughvuiteQ2TP3}
		\minterms{6}
		\maxterms{0,1,2,3}
		\indeterminats{5,4,7}
		\implicant{4}{6}{orange}
		\end{KarnaughvuiteQ2TP3}
	\end{center}
	\caption{$Z$: Salida - Máq. de Moore} 
	\label{2_fig6}
\end{figure}

Tomando las agrupaciones marcadas en cada mapa podemos formas las ecuaciones que representarán a  $Q_{0_{t+1}}, Q_{1_{t+1}}, Q_{2_{t+1}}$ y $Z$ respectivamente, quedando las ecuaciones de la siguiente manera:

\begin{align*}
	Q_{0_{t+1}} &= W * \overline{Q_{1}} + W * ( Q_{2} + Q_{0} ) \\
	Q_{1_{t+1}} &= W * ( Q_{2} + Q_{0} ) + Q_{1} * ( W + Q_{0} ) \\
	Q_{2_{t+1}} &= W * \overline{Q_{2}} * Q_{1} * \overline{Q_{0}} \\
	Z &= Q_{2} \\
\end{align*}
Una vez que ya tenemos las ecuaciones procedemos a armar el circuito, una simulación por verilog y realizar las mediciones para ver el funcionamiento de la máquina de estado.


\subsection{Mealy}

También se nos pidi\'o armar una m\'aquina pero de Mealy en lugar de Moore. Para esto rediseñamos el diagrama de la máquina de estados utilizando la notaci\'on de Mealy con flechas que indiquen la entrada y la salida entre transición de estados. Utilizando este método tendremos menos estados pero la salida esta vez será dependiente tanto de los estados como de las entradas.

\begin{figure}[H]
	\centering
	\includegraphics[scale=0.6]{Imagenes/diagestmealy.jpg}
	\caption{Diagrama de Estados - Máquina de Mealy}
	\label{2_fig7}
\end{figure}

Partiendo de este nuevo diagrama, se tabula un nuevo cuadro con la transiciones de estados y la salida, para luego asignarle un valor numérico a cada estado. Como esta vez son 4 estados solo se necesitara representar desde el 0 al 3, los cuales en binario serán 0-0 y 1-1, solo se necesitaran 2 flip flops para esta máquina de estados. La siguiente tabla muestra los valores que $Q_{1_{t+1}}$ $Q_{0_{t+1}}$ y $Z$ tendrán ahora.

\begin{figure}[H]
	\begin{center}
		\begin{tabular}{|c|c|c|c|c|}
\hline
\multirow{2}{*}{\textbf{\begin{tabular}[c]{@{}c@{}}Estado\\ Actual\end{tabular}}} & \multicolumn{2}{c|}{\textbf{\begin{tabular}[c]{@{}c@{}}Estado\\ Siguiente\end{tabular}}} & \multicolumn{2}{c|}{\textbf{Salida: Z}} \\ \cline{2-5} 
 & \textbf{W = 0} & \textbf{W = 1} & \textbf{W = 0} & \textbf{W = 1} \\ \hline
\textbf{A$_{000}$} & \textbf{A} & \textbf{B} & \textbf{0} & \textbf{0} \\ \hline
\textbf{B$_{001}$} & \textbf{A} & \textbf{C} & \textbf{0} & \textbf{0} \\ \hline
\textbf{C$_{010}$} & \textbf{D} & \textbf{C} & \textbf{0} & \textbf{0} \\ \hline
\textbf{D$_{011}$} & \textbf{A} & \textbf{B} & \textbf{0} & \textbf{1} \\ \hline
		\end{tabular}
	\caption{Transiciones - Máquina de Mealy} 
	\label{2_fig8}
	\end{center}
\end{figure}


En este caso hicimos un análisis previo y decidimos ordenar los estados por código de Grey para q de un estado a otro solo haya un cambio de estado, solo un bit cambiaría entre cada transición excepto del estado C al A, lo que significaría que un solo flip flop cambiaría entre estado. Siendo entonces el estado A la combinación 0-0, B la combinación 0-1, C la combinación 1-1 y por último D la combinación 1-0. Utilizamos este método ya que nos ahorraba un integrado a comparación del anterior. La siguiente tabla muestra las transiciones con el código de Grey:


\begin{figure}[H]
	\begin{center}
		\begin{tabular}{|c|c|c|c|c|c|c|c|}
\hline
\multicolumn{2}{|c|}{\multirow{2}{*}{\textbf{\begin{tabular}[c]{@{}c@{}}Estado\\ Actual\end{tabular}}}} & \multicolumn{4}{c|}{\textbf{Estado Siguiente}} & \multicolumn{2}{c|}{\multirow{2}{*}{\textbf{Salida: Z}}} \\ \cline{3-6}
\multicolumn{2}{|c|}{} & \multicolumn{2}{c|}{\textbf{W = 0}} & \multicolumn{2}{c|}{\textbf{W = 1}} & \multicolumn{2}{c|}{} \\ \hline
\textbf{$Q_{1_{t}}$} & \textbf{$Q_{0_{t}}$} & \textbf{$Q_{1_{t+1}}$} & \textbf{$Q_{0_{t+1}}$} & \textbf{$Q_{1_{t+1}}$} & \textbf{$Q_{0_{t+1}}$} & \textbf{W = 0} & \textbf{W = 1} \\ \hline
\textbf{0} & \textbf{0} & \textbf{0} & \textbf{0} & \textbf{0} & \textbf{1} & \textbf{0} & \textbf{0} \\ \hline
\textbf{0} & \textbf{1} & \textbf{0} & \textbf{0} & \textbf{1} & \textbf{1} & \textbf{0} & \textbf{0} \\ \hline
\textbf{1} & \textbf{1} & \textbf{1} & \textbf{0} & \textbf{1} & \textbf{1} & \textbf{0} & \textbf{0} \\ \hline
\textbf{1} & \textbf{0} & \textbf{0} & \textbf{0} & \textbf{0} & \textbf{1} & \textbf{0} & \textbf{1} \\ \hline
		\end{tabular}
	\caption{Transiciones con Flip Flop - Máquina de Mealy} 
	\label{2_fig9}
	\end{center}
\end{figure}

Teniendo en cuenta que la única diferencia en esta parte con respecto a Moore es que la salida $Z$ depende de la entrada y los estados, pasamos a analizar las columnas de la tabla anterior. Las resolvimos con mapas de Karnaugh para simplificar más rápido los minit\'erminos quedando los estados y la salida de las siguientes maneras:

\begin{figure}[H]
	\begin{center}
		\begin{KarnaughvuiteTP3}
		\minterms{4,5,6}
		\maxterms{0,1,3,2,7}
		%\indeterminats{2,5}
		\implicant{4}{5}{green}
		\implicant{6}{4}{red}
		%\implicant{7}{7}{blue}
		\end{KarnaughvuiteTP3}
	\end{center}
	\caption{$Q_{0_{t+1}}$ Máq. de Mealy} 
	\label{2_fig10}
\end{figure}

\begin{figure}[H]
	\begin{center}
		\begin{KarnaughvuiteTP3}	
		\minterms{2,5}
		\maxterms{0,1,3,4,6,7}
		%\indeterminats{2,5}
		\implicant{2}{2}{green}
		\implicant{5}{5}{red}
		\end{KarnaughvuiteTP3}
	\end{center}
	\caption{$Q_{1_{t+1}}$ Máq. de Mealy} 
	\label{2_fig11}
\end{figure}
	
\begin{figure}[H]
	\begin{center}
		\begin{KarnaughvuiteTP3}
		\minterms{7}
		\maxterms{0,1,2,3,4,5,6}
		%\indeterminats{2,5}
		\implicant{7}{7}{green}
		\end{KarnaughvuiteTP3}
	\end{center}
	\caption{$Z$: Salida - Máq. de Mealy} 
	\label{2_fig12}
\end{figure}

\end{document}
