\documentclass[10pt,a4paper]{article}
\usepackage[utf8]{inputenc}
\usepackage[spanish]{babel}
\usepackage{amsmath}
\usepackage{amsfonts}
\usepackage{amssymb}
\usepackage{graphicx}
\usepackage{float}
\usepackage[left=2cm,right=2cm,top=2cm,bottom=2cm]{geometry}
\usepackage{circuitikz}
\usepackage{tikz}
\usepackage{tikz-timing}
\usetikztiminglibrary[rising arrows]{clockarrows}
\usepackage{pgfplots}
\pgfplotsset{compat=1.15}
\usetikzlibrary{matrix,calc}
\usepackage{listings}
\usepackage{pdfpages}
\usepackage{lscape}
\usepackage{multicol}


\input{karnaugh_def_tp3_ej1_moore}

\begin{document}
\part*{Ejercicio 1}
Se nos pidió diseñar e implementar una maquina de estados capaz de controlar la activación de dos bombas B1 y B2 (simulando su encendido con un LED) que deben mantener el nivel de agua de un depósito que dispone de dos sensores S e I, colados en la parte superior e inferior del depósito, respectivamente. La operatoria de la máquina sigue las siguientes reglas:

\begin{itemize}
\item Si el agua ha superado un sensor, su valor de salida será: 1.
\item Si el deposito estuviera lleno (I=S=1) no se activaría ninguna bomba.
\item Si el deposito estuviera vacío (I=S=0) se activarían ambas bombas.
\item Si el deposito estuviera lleno por la mitad (I=1, S=0) se activaría la ultima bomba en no activarse. 
\end{itemize}

Se nos solicitó tener en cuenta las siguientes consideraciones para la implementación de la maquina de estados:

\begin{itemize}
\item Implementar la solución utilizando tanto una maquina de Mealy como de Moore.
\item Muestre claramente el diagrama de estados y transiciones.
\item Respaldar el diseño con una simulación en Verilog.
\end{itemize} 

\section*{Implementación Maquina de Moore}
\subsection*{Diagrama de Estados y Transiciones}

Se comenzó por la implementación de la máquina de estados de Moore, para ello, se comenzó por realizar un diagrama de estados y transiciones que describa la maquina de estados. Para lo cual se listará primero a las entradas, estados y salidas posibles, y se hará una breve descripción para una mejor comprensión del diagrama.

\subsubsection*{Entradas:}

\begin{description}
\item[Vacío:] Configuración de entrada S=0, I=0, que indica que el depósito se encuentra vacío.
\item[Medio:] Configuración de entrada S=0, I=1, que indica que el depósito esta lleno por la mitad.
\item[Lleno:] Configuracion de entrada S=1, I=1, que indica que el depósito se encuentra lleno.
\end{description}

\subsubsection*{Estados:}

\begin{description}
\item[Ninguna:] Estado que indica que ninguna bomba esta encendida.
\item[Una Sola:] Estado que indica que solo una bomba se encuentra encendida.
\item[Ambas:] Estado que indica que ambas bombas se encuentran encendidas.
\end{description}

\subsubsection*{Salidas:}

\begin{description}
\item[$b_1=0$ y $b_2=0$:] Esta configuración de salida no enciende ninguna bomba.
\item[$b_1=1$,$b_2=0$ o $b_1=0$,$b_2=1$:] Solo se enciende una de las bombas que controla el circuito.
\item[$b_1=1$ y $b_2=1$:] Esta configuración de salida enciende las dos bombas.
\end{description}

A continuación se presenta el diagrama de estados y transiciones, a partir del cual se diseñó la maquina de estados:

\begin{figure}[H]
\centering
\includegraphics[scale=0.4]{images/diagrama_estados_moore.png}
\caption{Diagrama de estados y transiciones} \label{1_figa}
\end{figure}

Al contar con 3 estados diferentes, mi maquina de estados necesitará como mínimo dos Flip-Flop's para almacenar el estado actual. Los tres estados posibles se codifican a traves de dos variables $y_1$ e $y_0$, de esta forma, tendremos las siguientes configuraciones:

\bigskip

\begin{table}[ht]
	\centering
	\begin{tabular}{c|c}
	Estado & $y_1y_0$ \\ 
	\hline 
	Ninguna & 00 \\ 
	Una Sola & 01 \\ 
	Ambas & 11 \\ 
	\end{tabular} 
\end{table}


Habiendo definido los estados del diseño, se procedió a completar la correspondiente tabla de asignación de estados:
\bigskip
\begin{table}[ht]
	\centering
	\begin{tabular}{c|c|c|c|c}
	Estado Actual & \multicolumn{3}{c|}{Próximo Estado$(Y_1Y_0)$} & Salida\\
	\cline{2-4}
	$(y_1y_0)$ & $SI=00$ & $SI=01$ & $SI=11$ & $(b_1b_2)$\\
	\hline
	00 & 11 & 01 & 00 & 00 \\
	01 & 11 & 01 & 00 & 01 o 10 (Alternado) \\
	11 & 11 & 01 & 00 & 11 \\
	\end{tabular}
	\caption{Tabla de asignación de estados}
	\label{1_t1}
\end{table}

A partir del Cuadro \ref{1_t1} se confeccionó el Cuadro \ref{1_t2}, que implementa la tabla de verdad que determina el próximo estado ($Y_1Y_0$) en función del estado anterior ($y_1y_0$) y las entradas ($SI$). 



\begin{table}[H]
	\centering
	\begin{tabular}{c|c|c}
	$y_1y_0SI$ & $Y_1$ & $Y_0$  \\ 
	\hline 
	0000 & 1 & 1  \\ 
	\hline 
	0001 & 0 & 1  \\ 
	\hline 
	0010 & x & x  \\ 
	\hline 
	0011 & 0 & 0  \\ 
	\hline 
	0100 & 1 & 1  \\ 
	\hline 
	0101 & 0 & 1  \\ 
	\hline 
	0110 & x & x  \\ 
	\hline 
	0111 & 0 & 0  \\ 
	\hline 
	1000 & x & x  \\ 
	\hline 
	1001 & x & x  \\ 
	\hline 
	1010 & x & x  \\ 
	\hline 
	1011 & x & x  \\ 
	\hline 
	1100 & 1 & 1  \\ 
	\hline 
	1101 & 0 & 1  \\ 
	\hline 
	1110 & x & x  \\ 
	\hline 
	1111 & 0 & 0  \\ 
	\end{tabular} 
	\caption{Tabla de verdad cambio de estado}
	\label{1_t2}
\end{table}

Se determinó que es estado $SI=10$ no es una combinación posible, ya que indicaría que el agua ha superado el sensor superior pero no el inferior, lo cual es incompatible con el modelo del deposito, e indicaría un error en los sensores. Como el manejo de errores en el sensores excede los requisitos de la consigna es que se determinó que no son combinaciones posibles y las salidas correspondientes a estas configuraciones se determinaron como 'don't care'

La simplificación mediante Mapas de Karnaugh arrojó los siguientes resultados:

\begin{multicols}{2}


\begin{center}
\begin{Karnaugh}
   \minterms{0,4,12}
   \maxterms{1,3,5,7,13,15}
   \indeterminats{2,6,8,9,10,11,14}
   \implicantcostats{0}{10}{}
\end{Karnaugh}
\[
Y_1 = \overline{I}
\]

\end{center}


\columnbreak

\begin{center}
\begin{Karnaugh}
   \minterms{0,1,4,5,12,13}
   \maxterms{3,7,15}
   \indeterminats{2,6,8,9,10,11,14}
   \implicant{0}{9}{}
\end{Karnaugh}
\end{center}

\[
Y_0 = \overline{S}
\]
\end{multicols}



Se observa que el 'próximo estado' no depende del estado actual, sino solamente de las entradas.

\subsection*{Implementación}

La lógica que determina la transición de un estado a otro está dada por las expresiones de $Y_1$ e $Y_0$, desarrolladas en el apartado anterior. Se determinó adecuada la utilización de Flip-Flop's D para almacenar las variables $y_1$  e $y_0$ que determinan el estado de la máquina. La implementación del circuito se muestra en la Figura \ref{1_fig4}

\begin{figure}[H]
\centering
\includegraphics[scale=0.3]{images/input_moore.png}
\caption{Lógica de transición de estados}
\label{1_fig4}
\end{figure}

Para implementar la lógica combinacional de las salidas en función de los estados actuales, se decidió utilizar un multiplexor, de forma que por un lado se implemente la lógica que controle las salidas cuando éstas son iguales (estados \emph{Ninguna} y \emph{Ambas}) y por otro la lógica cuando las salidas deben ser complementarias (estado \emph{Una Sola}).

Para decodificar el estado de la máquina implementada, y poder controlar el mutiplexor mencionado y la lógica combinacional de las salidas, se utilizó un decodificador  2-a-4 74HC139. Este integrado implementa un decodificador de salidas active LOW, de esta forma, puedo decodificar el estado actual de la máquina para el posterior uso en la lógica de salida.

Cuando la máquina se encuentra en el estado \emph{Una Sola}(\emph{Una Sola=1}), las salidas no solo son complementarias (una bomba está encendida y la otra apagada), sino que además cada vez que 'entro' y 'salgo' del estado en cuestión, las salidas se deben intercalar, como indica la consigna. Para lograr esto, se utilizó un Flip-Flop JK en modo Toggle (J=1, K=1). El Clock del Flip-Flop en cuestión es alimentado por la señal \emph{~(Una Sola)}, como indica la Figura \ref{1_fig3}.

\begin{figure}[H]
\centering
\includegraphics[scale=0.3]{images/jk_toggle_moore.png}
\caption{Flip-Flop JK para alternar salidas}
\label{1_fig3}
\end{figure}

Las salidas de este Flip-Flop alimentaron las entradas '0' de dos multiplexores como indica la Figura \ref{1_fig2}. De esta forma cuando la máquina se encuentre en el estado \emph{Una Sola}, las salidas $b_1$ y $b_2$ serán complementarias


\begin{figure}[H]
\centering
\includegraphics[scale=0.3]{images/multiplexion_moore.png}
\caption{Multiplexación a la salida}
\label{1_fig2}
\end{figure}

Por otro lado, cuando la máquina de estados se encuentre en el estado \emph{Ambos}, se seleccionará la entrada '1' del multiplexor, y obtendré un 1 a la salida de los mutiplexores cuando \emph{Ambos=0} y un 0 cuando  \emph{Ambos=1}. Como las salidas $b_1$ y $b_2$ son las salidas de los multiplexores negadas, entonces obtendremos $b_1b_2=11$ cuando \emph{Ambas=1} y $b_1b_2=00$ cuando \emph{Ambas=0}

\subsubsection*{Simulación Verilog}

La Figura \ref{1_fig5} muestra el diagrama temporal generado por el código de Verilog presentado en el Anexo. 

\begin{figure}[H]
\centering
\includegraphics[scale=0.45]{images/time_diagram_moore.png}
\caption{Diagrama temporal de la simulación en Verilog}
\label{1_fig5}
\end{figure}


\section*{Implementación Maquina de Mealy}
\subsection*{Diagrama de Estados y Transiciones}
Al igual que en el proceso de diseño de la máquina de estados de Moore, el primer paso fue determinar los estados 

\end{document}